\documentclass[11pt]{article}
\usepackage{geometry}                
\geometry{letterpaper}                   

\usepackage{graphicx}
\usepackage{amssymb}
\usepackage{epstopdf}
\usepackage{natbib}
\usepackage{amssymb, amsmath}
\DeclareGraphicsRule{.tif}{png}{.png}{`convert #1 `dirname #1`/`basename #1 .tif`.png}

%\title{Title}
%\author{Name 1, Name 2}
%\date{date} 




%Additinal packages ants in the pants

\usepackage[]{algorithm2e}
\usepackage{todonotes}
\usepackage[german]{babel}
\usepackage[utf8]{inputenc}	










\begin{document}



\thispagestyle{empty}

\begin{center}
\includegraphics[width=5cm]{ETHlogo.eps}

\bigskip


\bigskip


\bigskip


\LARGE{ 	Lecture with Computer Exercises:\\ }
\LARGE{ Modelling and Simulating Social Systems with MATLAB\\}

\bigskip

\bigskip

\small{Project Report}\\

\bigskip

\bigskip

\bigskip

\bigskip


\begin{tabular}{|c|}
\hline
\\
\textbf{\LARGE{Dessert ant Adaptive orientation of dessert ants }}\\
\textbf{\LARGE{(lateinischer Name)}}\\
\\
\hline
\end{tabular}
\bigskip

\bigskip

\bigskip

\LARGE{Florian Hasler, Matthias Heinzmann,
\\ Andreas Urech,  Dominik Werner}



\bigskip

\bigskip

\bigskip

\bigskip

\bigskip

\bigskip

\bigskip

\bigskip

Zurich\\
December 2015\\

\end{center}



\newpage

%%%%%%%%%%%%%%%%%%%%%%%%%%%%%%%%%%%%%%%%%%%%%%%%%

\newpage
\section*{Agreement for free-download}
\bigskip


\bigskip


\large We hereby agree to make our source code for this project freely available for download from the web pages of the SOMS chair. Furthermore, we assure that all source code is written by ourselves and is not violating any copyright restrictions.

\begin{center}

\bigskip


\bigskip

\begin{tabular}{@{}p{3.3cm}@{}p{6cm}@{}@{}p{6cm}@{}}
\begin{minipage}{3cm}

\end{minipage}
&
\begin{minipage}{6cm}
\large Florian Hasler
\end{minipage}
&
\begin{minipage}{6cm}
\large Matthias Heinzmann
\end{minipage}
\end{tabular}
\ \\
\ \\
\ \\
\ \\
\ \\
\ \\
\ \\
\begin{tabular}{@{}p{3.3cm}@{}p{6cm}@{}@{}p{6cm}@{}}
\begin{minipage}{3cm}
\end{minipage}
&
\begin{minipage}{6cm}
\large Andreas Urech
\end{minipage}
&
\begin{minipage}{6cm}
\large Dominik Werner
\end{minipage}
\end{tabular}
\end{center}
\newpage

%%%%%%%%%%%%%%%%%%%%%%%%%%%%%%%%%%%%%%%



% IMPORTANT
% you MUST include the ETH declaration of originality here; it is available for download on the course website or at http://www.ethz.ch/faculty/exams/plagiarism/index_EN; it can be printed as pdf and should be filled out in handwriting


%%%%%%%%%% Table of content %%%%%%%%%%%%%%%%%

\tableofcontents

\newpage

%%%%%%%%%%%%%%%%%%%%%%%%%%%%%%%%%%%%%%%



\section{Abstract}
The Desert Ant (Cataglyphis) is an interesting creature. It lives in the desert where conditions are very harsh. Being to long outside of its nest the ant will die due to the ambient heat, therefore it is more than vital for the ant's survival that it has a sophisticated way to navigate through the pans of sand that make out most of its habitat. Since the early 20th century biologists have been fascinated by this ant and its navigation through the desert. Recently there has been speculation about the methods that are utilized by these ants to navigate between sources of food and their nest. If we could reproduce the path-pattern observed by real ants with a model we would be able to predict how they would react if something would change in their ecosystem.
We will base our work upon an essays written by R. Wehner \cite{Wehner2003},\cite{Wehner1988}, \cite{Wehner1998}.\\
 In his work he suggests 3 methods of navigation which the ants can use, namely pathintegration, pheromones and visual landmark recognition. In his final conclusion he considers all 3 methods important and assumes that the ant has a way of adopting the priority of each mechanism according to the current situation.
In Previous years a group called gordonteam\cite{GordonTeam2008} already implemented a model based on the before mentioned article. In the report they noted that their path integration did not work and that they were not able to combine all 3 methods. Our plan is to extend upon their work and fix the problems as well as to bind all 3 methods together. Furthermore we want to implement a memory system along with a learning machine which will allow the ant to dynamically adapt to an environment.




\newpage
\section{Individual contributions}
\subsection{Orientation}
As mentioned the desert ant's main means of orientation are:
\begin{itemize}
\item pathintegration
\item local landmark orientation
\item pheromones
\end{itemize}
The pathintegration and the local landmark orientation are primarily used in order to get back to the nest from a food source, whereas the orientation by pheromones is mainly used to find a prelocated food source. \todo{in an educated guess check. Email an R. Wehner.}

\subsubsection{Pathintegration}
No matter the zig-zag way out of its nest, the ant is able to return in a more or less straight line. This remarkable ability is reached by pathintegration. The ant iteratively computes the mean of all its turning angles executed and is therefore always aware in which direction its nest is located. These calculations are executed with imperfections which accumulate with growing distance. In extreme cases pathintegration causes the ant to miss its nest and other means of orientation are necessary. \\
In\footnote{Wehner1988\cite{Wehner1988} Page 5288} the following formulas have been derived, they take into account the rough approximation of the ants pathintegration. 
\begin{align*}
\varphi_{n+1} =& \varphi +k \cdot \frac{(180^{\circ} + \delta)\cdot(180^{\circ} - \delta)\cdot\delta}{l_{n}} \\
l_{n+1} =& l_{n}+ 1 - \frac{\delta}{90^{\circ}}
\end{align*}
\begin{tabbing}
\hspace{1cm}\=\kill
$l_{n}:$ \> current distance (in unitlength) to the nest \\ 
$\varphi_{n}:$ \> current angle that points backwards to the nest \\ 
$k:$ \>  normalization factor\\ 
$\delta:$ \>  executed angle\\ 
\end{tabbing} 
$l$ and $\varphi$ are constantly updated and together they form the \textit{global vector}.
\todo{damit nicht grösser als 90} 
\subsubsection{Local landmark orientation }
The ants can make use of landmark orientations. They can recreate a certain route by memorizing given landmarks in the correct sequence.

\subsubsection{Pheromone orientation}
Having found a food source the secretion becomes more intense. Other ants are attracted to the pheromones of their conspecifics and are therefore guided to the foodsource and thereby pheromone trace is further augmented. Once the foodsource is depleted the ants fan out and the pheromone evaporates exponentially \footnote{GordonTeam2008 \cite{GordonTeam2008} Page 9}.

 \begin{align*}
 I(t)=I(t - \Delta t) \exp (\frac{\log (\frac{1}{2}) \Delta  t}{t_{c}})\Delta t 
 \end{align*}
 
 
 
 %$I(t)=I(t - \Delta t) \exp (\frac{a}{b})\Delta t $










\subsubsection{Combination}
As mentioned the pheromone orientation is used as a mean to find a food source not a mean to return back to their nest.\todo{Nochmals klären ob das Stimmt.}
It is not that the various means of orientations are combined in a weighted fashion. Given certain criteria the different means are used separately.\\
The results gained R. Wehner's experiments indicate that ants use predominantly pathintegration, but if available using local landmarks is more preferable to using the global vector-navigation via pathintegration. It has been concluded that the ant transiently inhibits the global vector when being in familiar territory. The local vector only reemerges after the local vector ceases. Whilst navigating via the local vector the global vector is continuously updated.
\\ \

\begin{algorithm}[H]

  \SetKwFunction{algo}{ReturnToMyNest}
  \SetKwProg{myalg}{Algorithm}{}{}
  \myalg{\algo{}}{




%\SetKwFunction{goToMyNest}{goToMyNest}
%\goToMyNest
 \While{not at nest}{ 
 secrete pheromones; \\
 execute global vector;\\
 update global vector;
 
  \If{local vector recognised}{
   		\While{local vector $ > 0$}{
   		 execute local vector\;
   		 update local vector\;
   		 update global vector;
   			}
		}
}
 \Return
 } 
\caption{Returning to the nest}
\end{algorithm}















\newpage
\section{Introduction and Motivations}

\section{Description of the Model}

\section{Implementation}

\section{Simulation Results and Discussion}

\section{Summary and Outlook}

\section{References}
%-------------------------------------Bibliographie-------------------------------------
\bibliography{./Literatur}  
 %Zuerst muss man TexFile2 mal in ein PDF kompilieren, dann einmal Bibfile (F11), dann nochmal in ein PDF
 %\bibliographystyle{apalike}
  %\bibliographystyle{plain}
\bibliographystyle{abbrv}
\nocite{Wehner2008}
\nocite{Wehner2003}
\nocite{Wehner1988}
\nocite{GordonTeam2008}
%\nocite{Gebert2001}
\thispagestyle{plain}






\end{document}  



 
